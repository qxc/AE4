%   COMMANDS ZUM ZUSAMMENBAUEN DER KARTEN
%   ---------------------------------------

\newcommand{\And}{\textbf{AND} \newline}
\newcommand{\Cast}{\textbf{Cast}}
\newcommand{\Persistent}{\textbf{PERSISTENT}}
\newcommand{\Discard}{\textbf{DISCARD}}
\newcommand{\To}{\textbf{TO~}}
\newcommand{\Power}{\textbf{POWER }}
\newcommand{\Immediately}{\textbf{IMMEDIATELY}}
\newcommand{\Evolve}{\textbf{EVOLVE}}
\newcommand{\BloodMagic}{\textbf{Blood Magic}}
\newcommand{\Money}{\includegraphics[height = .4cm]{money}}
\newcommand{\Or}{\textbf{OR}}
\newcommand{\Link}{\newline\textbf{Link:} Two spells with link may be prepped to the same breach.}
\newcommand{\Name}{\hspace{20mm}}
\newcommand{\Echo}{\textbf{Echo}} %(Cast this spell without discarding it, then cast it again.)}
\newcommand{\Poison}{\textbf{Poison} }%(When a poisoned enemy is dealt damage, deal 1 additional damage per poison and remove 1 poison token.) }
\newcommand{\Overkill}{\textbf{Overkill}}%(If this kills a minion, deal the remaining damage to the nemesis.)}
\newcommand{\Energized}{\textbf{Energized}}% If this deals more than its base damage, }
\newcommand{\Countdown}{\textbf{Countdown}}% If this deals more than its base damage, }
\newcommand{\Attach}{\textbf{Attach} }% If this deals more than its base damage, }
\newcommand{\Shadow}{\textbf{Shadow:}} %Place a Shadow token on a minion. Prevent all damage dealt to a minion with a shadow token. Players can reduce a shadowed minion's life by spending \Money{}.}
%   TikZ/PGF Settings f�r die Karten
\pgfmathsetmacro{\cardwidth}{6}
\pgfmathsetmacro{\cardheight}{8.5}
\pgfmathsetmacro{\imagewidth}{\cardwidth}
\pgfmathsetmacro{\imageheight}{0.75*\cardheight}
\pgfmathsetmacro{\stripwidth}{0.7}
\pgfmathsetmacro{\strippadding}{0.2}
\pgfmathsetmacro{\textpadding}{0.1}
\pgfmathsetmacro{\titley}{.94*\cardheight}
\pgfmathsetmacro{\titlex}{2.15}



%   Formen der einzelnen Kartenelemente/-bestandteile
\def\shapeCard{(0,0) rectangle (\cardwidth,\cardheight)}
\def\shapeLeftStripTop{(\strippadding,\cardheight/2+1) rectangle (\strippadding+\stripwidth,\cardheight-\strippadding-\strippadding-1)}
\def\shapeLeftStripBot{(\strippadding,-0.2) rectangle (\strippadding+\stripwidth,\cardheight/2-3)}
\def\shapeLeftStripShort{(\strippadding,\cardheight-\strippadding-1) rectangle (\strippadding+\stripwidth,\cardheight+0.2)}
\def\shapeRightStripShort{(\cardwidth-\stripwidth-\strippadding,\cardheight-\strippadding-1) rectangle (\cardwidth-\strippadding,\cardheight+0.2)}
\def\shapeTitleArea{(2*\strippadding+\stripwidth,\cardheight-\strippadding) rectangle (\cardwidth-2*\strippadding-\stripwidth,\cardheight-2*\stripwidth)}
\def\shapeContentArea{(2*\strippadding+\stripwidth,0.5*\cardheight) rectangle (\cardwidth+0.2,-0.2)}


%   Stylings f�r die Elemente definieren
\tikzset{
    %   runde Ecken f�r die Karten
    cardcorners/.style={
        rounded corners=0.2cm
    },
    %   runde Ecken f�r die "F�hnchen"
    elementcorners/.style={
        rounded corners=0.1cm
    }
}


\newcommand{\InvMoney}{\includegraphics[height = .6cm]{InvMoney}}
\newcommand{\Health}{\includegraphics[height = .4cm]{heart}}
\newcommand{\Shield}{\includegraphics[height = .4cm]{shield}}

%   Rahmen der Karte
%\newcommand{\cardborder}{
%    \draw[lightgray,cardcorners] \shapeCard;
%}

%   Hintergrund der Karte
\newcommand{\cardbackground}[1]{
		\clip[cardcorners] \shapeCard;
    \node at (\cardwidth/2, \cardheight/2) {\includegraphics[height = 9.5cm]{#1}};
}

\newcommand{\attentioncardbackround}[1]{
  \cardbackground{#1}
  % Add a pattern to the cards
  \draw[pattern=dots](0, \cardheight/2) rectangle (\cardwidth, \cardheight);
}

\newcommand{\cardtitle}[1]{
    \node[text width=2.75cm, below right] at (\titlex,\titley) {
        
       \shadowtext{\parbox{2.8cm}{\begin{spacing}{1.2}\textbf{\hspace{-.5cm}\uppercase{\Large #1}}\end{spacing}}}
        
    };
}
\newcommand{\cardstorycontent}[1]{
	\node[text width=4.7cm, below right] at (\stripwidth*1.2, .85*\cardheight) {\textrm{#1}};
}
\newcommand{\evolve}[1]{
	\node[text width=4.4cm, below right] at (\cardwidth/12, 0.52*\cardheight) {\footnotesize{#1}};
}
\newcommand{\cardcontent}[1]{
    %\node[text width=4.5cm] at (\cardwidth/2, 0.35*\cardheight) {#1};
		\node[text width=4.5cm, below right] at (\stripwidth*1.02, 0.535*\cardheight) {\textrm{#1}};
}

\newcommand{\tier}[1]{
    %\node[text width=4.5cm] at (\cardwidth/2, 0.35*\cardheight) {#1};
		\node at (\cardwidth/8, \cardheight/12) {\tiny{#1}};
}

\newcommand{\delay}[1]{
		\node at (\cardwidth*.9, \cardheight/8) {#1};
}
\newcommand{\cardimage}[1]{
	\node at (\cardwidth*.56, \cardheight*.67) {\includegraphics[height = 2.3cm]{#1}};
}

\newcommand{\cardprice}[1]{
		%\pgfmathsetmacro{\x}{(2+#1)/12}
		%\pgfmathsetmacro{\redvalue}{255-(45*\x*\x)}
		%\pgfmathsetmacro{\greenvalue}{255-(85*\x*\x)}
		%\pgfmathsetmacro{\bluevalue}{255*(1-\x)*(1-\x)}
		\pgfmathsetmacro{\x}{(#1)/10}
		\pgfmathsetmacro{\redvalue}{(1-\x)*(1-\x)*0+2*\x*(1-\x)*275+\x*\x*200}
		\pgfmathsetmacro{\greenvalue}{(1-\x)*(1-\x)*100+2*\x*(1-\x)*275+\x*\x*100}
		\pgfmathsetmacro{\bluevalue}{(1-\x)*(1-\x)*200+2*\x*(1-\x)*275+\x*\x*0}
		%\definecolor{currentcolor}{RGB}{\redvalue, \greenvalue,\bluevalue}
		\definecolor{currentcolor}{RGB}{255, 255,255}
    %\node at (\stripwidth*1.8,\cardheight-.8*\stripwidth) {\InvMoney};
		\node at (\stripwidth*1.7,\cardheight-1.1*\stripwidth) {\color{black} \textbf{\LARGE #1}}; %change black to currentcolor to get gradient back
}

\newcommand{\carddevcost}[1]{
		\node at (\cardwidth*.88,\cardheight-1.1*\stripwidth) {\color{black} \textbf{\small (#1)}};
}

\newcommand{\cardhp}[1]{
		\node[left] at (\cardwidth*.93,\cardheight*.78) {\textbf{\Large #1}\Health};
}
\newcommand{\cardshield}[1]{
		\node[left] at (\cardwidth*.92,\cardheight*.72) {\textbf{\Large #1}\Shield};
}

\newcommand{\sync}[1]{
	\node at (\cardwidth-2*\stripwidth-.20, \stripwidth) {v#1};
}

\newcommand{\cardborder}{
    \draw[black, cardcorners, ultra thick] \shapeCard;
}


