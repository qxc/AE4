\documentclass[landscape]{standalone}
\usepackage{tikz} 
\usepackage{DejaVuSansMono}
\renewcommand*\familydefault{\ttdefault} %% Only if the base font of the document is to be typewriter style
\usepackage[T1]{fontenc}
\usepackage[none]{hyphenat}

\pgfmathsetmacro{\boardwidth}{13}
\pgfmathsetmacro{\boardheight}{10}
\pgfmathsetmacro{\cardwidth}{6}
\pgfmathsetmacro{\cardheight}{8.5}
\pgfmathsetmacro{\squarewidth}{2.5}
\pgfmathsetmacro{\squareheight}{2.25}
\pgfmathsetmacro{\tw}{8cm}

\def\shapeCard{(-\boardwidth, -\boardheight) rectangle (\boardwidth,\boardheight)}

\newcommand{\Money}{\includegraphics[height = .4cm]{money} }
\newcommand{\Trash}{\includegraphics[height = .4cm]{trash} }
\newcommand{\Damage}{\includegraphics[height = .4cm]{damage} }
\newcommand{\Health}{\includegraphics[height = .4cm]{heart} }
\newcommand{\Or}{\textbf{OR}}

\newcommand{\cardborder}{
    \draw[black] \shapeCard;
}   

\newcommand{\startingPrep}[8]{
\clip \shapeCard;
\node[rotate = 90*#2-90 ] at (-\boardwidth, \boardheight-1.5) {\includegraphics[height = 8cm]{#1}};
\node[rotate = 90*#4-90 ] at (-\boardwidth/3, \boardheight-1.5) {\includegraphics[height = 8cm]{#3}};
\node[rotate = 90*#6-90 ] at (\boardwidth/3, \boardheight-1.5) {\includegraphics[height = 8cm]{#5}};
\node[rotate = 90*#8-90 ] at (\boardwidth, \boardheight-1.5) {\includegraphics[height = 8cm]{#7}};
}

\newcommand{\basicinfo}{
	\node[text width = \tw] at (\boardwidth*.47,.75) {
	\textbf{Turn Order:}\\
	\small{\begin{enumerate}
	\item Casting Phase: In any order: You may cast any of your prepped spells in opened breaches. You must cast any of your prepped spells in closed breaches.
	\item Main Phase: You may do the following in any order, any number of times
		\begin{enumerate}
		\item Play gem and relic cards
		\item Focus or open breaches
		\item Prep spells to breaches. 
		\item Buy cards
		\item Gain charges
		\end{enumerate}
	\item Draw Phase: Place cards played this turn in your discard pile in any order. Draw cards until five cards in hand.\\
	\end{enumerate}}
	};
	\draw (-\boardwidth,-\boardheight+1) rectangle (-\boardwidth+\cardwidth/2,-\boardheight+\cardheight+1);
	\node[rotate=90] at (-\boardwidth+1, -\boardheight+\cardheight/2+1) {\Huge Deck};
	\draw (\boardwidth,-\boardheight+1) rectangle (\boardwidth-\cardwidth/2,-\boardheight+\cardheight+1);
	\node[rotate=270] at (\boardwidth-1, -\boardheight+\cardheight/2+1) {\Huge Discard};
	\node at (-8, -4) {\LARGE{Life:}};
	\draw (-6,-5.1) rectangle (-6+\squarewidth*6,-5.1+\squareheight);
}

\newcommand{\notes}[1]{
	\node[text width=10.5cm, left] at (\squarewidth-.5, \squareheight*.1 - 1) {\textbf{Notes:} #1};
}

\newcommand{\name}[1]{
\node at (0, 4) {\LARGE{#1}};
\draw (-\squarewidth/2,3.5) rectangle (\squarewidth/2, 3.5-\squareheight);
\node[text width = 2 cm] at (0, 5-\squareheight) {\begin{center}Player \\ Number\end{center}};
}
\newcommand{\startingHand}[5]{
\pgfmathsetmacro{\handheight}{3}
\pgfmathsetmacro{\handwidth}{2.25}
\node at (-\boardwidth+2, \handheight+1) {Starting Hand:  };
\node[text width = 2cm] at (-\boardwidth+1.25, \handheight) {#1};
\node[text width = 2cm] at (-\boardwidth+1.25+\handwidth, \handheight) {#2};
\node[text width = 2cm] at (-\boardwidth+1.25+2*\handwidth, \handheight) {#3};
\node[text width = 2cm] at (-\boardwidth+1.25+3*\handwidth, \handheight) {#4};
\node[text width = 2cm] at (-\boardwidth+1.25+4*\handwidth, \handheight) {#5};
}


\newcommand{\startingDeck}[5]{
\pgfmathsetmacro{\deckheight}{1}
\pgfmathsetmacro{\deckwidth}{2.25}
\node at (-\boardwidth+2, \deckheight+1) {Starting Deck:  };
\node[text width = 2cm] at (-\boardwidth+1.25, \deckheight) {#1};
\node[text width = 2cm] at (-\boardwidth+1.25+\deckwidth, \deckheight) {#2};
\node[text width = 2cm] at (-\boardwidth+1.25+2*\deckwidth, \deckheight) {#3};
\node[text width = 2cm] at (-\boardwidth+1.25+3*\deckwidth, \deckheight) {#4};
\node[text width = 2cm] at (-\boardwidth+1.25+4*\deckwidth, \deckheight) {#5};
}
%
%\newcommand{\startingHand2}[5]{
%\pgfmathsetmacro{\handheight}{3}
%\pgfmathsetmacro{\handwidth}{1.5}
%\node at (-\boardwidth+2, \handheight+1) {Starting Hand:  };
%\node at (-\boardwidth+1, \handheight) {\includegraphics[height = 1cm]{#1}};
%\node at (-\boardwidth+1+\handwidth, \handheight) {\includegraphics[height = 1cm]{#2}};
%\node at (-\boardwidth+1+2*\handwidth, \handheight) {\includegraphics[height = 1cm]{#3}};
%\node at (-\boardwidth+1+3*\handwidth, \handheight) {\includegraphics[height = 1cm]{#4}};
%\node at (-\boardwidth+1+4*\handwidth, \handheight) {\includegraphics[height = 1cm]{#5}};
%}
%
%
%\newcommand{\startingDeck2}[5]{
%\pgfmathsetmacro{\deckheight}{1}
%\pgfmathsetmacro{\deckwidth}{1.5}
%\node at (-\boardwidth+2, \deckheight+1) {Starting Deck:  };
%\node at (-\boardwidth+1, \deckheight) {\includegraphics[height = 1cm]{#1}};
%\node at (-\boardwidth+1+\deckwidth, \deckheight) {\includegraphics[height = 1cm]{#2}};
%\node at (-\boardwidth+1+2*\deckwidth, \deckheight) {\includegraphics[height = 1cm]{#3}};
%\node at (-\boardwidth+1+3*\deckwidth, \deckheight) {\includegraphics[height = 1cm]{#4}};
%\node at (-\boardwidth+1+4*\deckwidth, \deckheight) {\includegraphics[height = 1cm]{#5}};
%}

\newcommand{\ability}[2]{

\node[text width = 16 cm] at (-1, -6.5) {\textbf{Ability -} #1 };

\ifnum 1>#2\relax\else
\node at (-9+\squarewidth/2, -10+\squareheight/2) {\includegraphics[height = 2cm]{chargetest}};
\fi

\ifnum 2>#2\relax\else
\node at (-6+\squarewidth/2, -10+\squareheight/2) {\includegraphics[height = 2cm]{chargetest}};
\fi

\ifnum 3>#2\relax\else
\node at (-3+\squarewidth/2, -10+\squareheight/2) {\includegraphics[height = 2cm]{chargetest}};
\fi

\ifnum 4>#2\relax\else
\node at (0+\squarewidth/2, -10+\squareheight/2) {\includegraphics[height = 2cm]{chargetest}};
\fi

\ifnum 5>#2\relax\else
\node at (3+\squarewidth/2, -10+\squareheight/2) {\includegraphics[height = 2cm]{chargetest}};
\fi

\ifnum 6>#2\relax\else
\node at (6+\squarewidth/2, -10+\squareheight/2) {\includegraphics[height = 2cm]{chargetest}};
\fi

\ifnum 7>#2\relax\else
\node at (9+\squarewidth/2, -10+\squareheight/2) {\includegraphics[height = 2cm]{chargetest}};
\fi
}