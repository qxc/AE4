\nonstopmode
\documentclass{article}
%   VORAUSGESETZTE LIBS
%   ------------------

%   R�nder des Dokuments anpassen
\usepackage[margin=6mm,top=5mm]{geometry}

%   Sharten Faken der aufen kaken
\usepackage[none]{hyphenat}

%   Schriftart der auf den Karten eingesetzten Texte
\usepackage{anttor}

%   UTF-8 Encoding der TeX-Dateien
\usepackage[utf8]{inputenc}

%   deutsches Sprachpaket
\usepackage[german]{babel}

%   optischer Randausgleich
\usepackage{microtype}

%   Einbinden von Grafiken
%\usepackage{graphicx}

%   Definieren und Verwenden von Farben
\usepackage{color}

%   TikZ zum "Malen" von Grafiken, in diesem Falle f�r die Karten
\usepackage{tikz}
\usetikzlibrary{patterns}
\usetikzlibrary{shadows}

%   Symbole dazuladen; Verwendung \ding{<nummer>}
\usepackage{pifont}
%   weitere Symbole
\usepackage{fourier-orns}
%\usepackage{textcomp}
%\usepackage{mathcomp}


\usepackage{DejaVuSansMono}
\renewcommand*\familydefault{\ttdefault} %% Only if the base font of the document is to be typewriter style
\usepackage[T1]{fontenc}

\usepackage{xcolor}
\usepackage{shadowtext}

\usepackage{setspace}
\usepackage{mwe}
%   FARBEN DER ELEMENTE/BESTANDTEILE DER KARTEN
%   -----------------------------------------

%   Hintergrundfarbe f�r den Titel-Kasten
    \definecolor{titlebg}{RGB}{30,30,30}

%   Farben der "F�hnchen" zur Kennzeichnung der unterschiedlichen Kartentypen
    \definecolor{defensebg}{RGB}{0,100,200}
    \definecolor{positivebg}{RGB}{80,180,0}
    \definecolor{negativebg}{RGB}{200,50,50}
    \definecolor{cannonbg}{RGB}{100,100,100}
		\definecolor{actionbg}{RGB}{0,0,255}
		\definecolor{attackbg}{RGB}{255,128,0}
		\definecolor{moneybg}{RGB}{230,180,0}
		%\definecolor{itembg}{RGB}{230,180,0}

%   Farbe des "F�hnchens" zur Angabe des Preises der Karten
    \definecolor{pricebg}{RGB}{230,180,0}

%   Hintergrundfarbe f�r den Textbereich
    %\definecolor{content}{RGB}{250,250,245}
    \definecolor{contentbg}{RGB}{255,255,255}
\usepackage[none]{hyphenat}
\pgfmathsetmacro{\cardwidth}{17.5}
\pgfmathsetmacro{\cardheight}{26}
\pgfmathsetmacro{\offset}{.5}
\pgfmathsetmacro{\tw}{8cm}
\pgfmathsetmacro{\squarewidth}{2.5}
\pgfmathsetmacro{\squareheight}{2.25}
\def\shapeCard{(0,0) rectangle (\cardwidth,\cardheight)}
\newcommand{\Money}{\includegraphics[height = .4cm]{money} }
\newcommand{\cardborder}{\draw[black] \shapeCard;
\draw[ultra thick] (\cardwidth/2, \cardheight/2) rectangle (0, \cardheight);}
\newcommand{\name}[1]{\node[text width = \tw] at (\cardwidth/4, \cardheight-\offset) {\LARGE{#1}};}
\newcommand{\setup}[1]{\node[text width = \tw] at (\cardwidth/4, \cardheight*.93) {\textbf{Setup:} #1};}
\newcommand{\unleash}[1]{\node[text width = \tw] at (\cardwidth/4, \cardheight*.74) {\textbf{Unleash:} #1};}
\newcommand{\difficulty}[1]{\node[text width = \tw] at (\cardwidth/4, \cardheight*.6) {\textbf{Extra difficulty:} #1};}
\newcommand{\rules}[1]{\node at (\cardwidth*.7, \cardheight/2+10) ; \node[text width = 7 cm] at (\cardwidth*3/4, \cardheight/2+6) {#1};}
\newcommand{\image}[1]{\includegraphics[height = .8cm]{#1}}
\newcommand{\hp}[1]{\textbf{#1} \image{heart}}
\begin{document}
\begin{center}
\pagestyle{empty}\begin{tikzpicture} 
\draw[black] \shapeCard; 
ewline\newcommand{\Or}{\newline \textbf{OR} \newline};
\draw[ultra thick] (\cardwidth/2, \cardheight*.56) rectangle (0, \cardheight); 
\name{Umbra Titan \image{umbra} \hp{70}};
\setup{Place 8 Burrow counters on this mat.}
\unleash{If there are two nemesis turn order cards in the turn order discard pile, Gravehold suffers 2 damage \Or remove a Burrow counter. \newline \newline \newline Otherwise, Any player suffers 2 damage \Or remove a Burrow counter.}
\rules{If the nemesis ever has 0 Burrow counters, the players lose. };
\end{tikzpicture}

\begin{tikzpicture} 
\draw[black] \shapeCard; 
ewline\newcommand{\Or}{\newline \textbf{OR} \newline};
\draw[ultra thick] (\cardwidth/2, \cardheight*.56) rectangle (0, \cardheight); 
\name{Barrage Man \image{barrage} \hp{70}};
\setup{}
\unleash{Gain a Barrage token}
\difficulty{sadfg}
\rules{Each breach mage is at a position close/m/far, deal/take more damage up close/ reduced at distance, nemesis has different effects based on position. Unleash with 3 effects based on position mb? Different costs to go from one position to another. Discard cards, spend aether, discard spells. passive effects at different ranges. The 'party' is represented by a single mini. End of turn, 3 or more barrage tokens triggers different effect based on position. At the start of your casting phase, you may discard 1 card to move position. Short distance: Spells deal +2 damage, Medium 0 damage, Far -2 damage. Barrage effect: Short: Any player suffers 5 damage. Medium: Any player suffers 2 damage. Long: Gravehold suffers 5 damage.};
\end{tikzpicture}


 \end{center} 
 \end{document}