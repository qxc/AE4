%   COMMANDS ZUM ZUSAMMENBAUEN DER KARTEN
%   ---------------------------------------

%   TikZ/PGF Settings f�r die Karten
\pgfmathsetmacro{\cardwidth}{6}
\pgfmathsetmacro{\cardheight}{8.5}
\pgfmathsetmacro{\imagewidth}{\cardwidth}
\pgfmathsetmacro{\imageheight}{0.75*\cardheight}
\pgfmathsetmacro{\stripwidth}{0.7}
\pgfmathsetmacro{\strippadding}{0.2}
\pgfmathsetmacro{\textpadding}{0.1}
\pgfmathsetmacro{\titley}{.94*\cardheight}
\pgfmathsetmacro{\titlex}{2.3}



%   Formen der einzelnen Kartenelemente/-bestandteile
\def\shapeCard{(0,0) rectangle (\cardwidth,\cardheight)}
\def\shapeLeftStripTop{(\strippadding,\cardheight/2+1) rectangle (\strippadding+\stripwidth,\cardheight-\strippadding-\strippadding-1)}
\def\shapeLeftStripBot{(\strippadding,-0.2) rectangle (\strippadding+\stripwidth,\cardheight/2-3)}
\def\shapeLeftStripShort{(\strippadding,\cardheight-\strippadding-1) rectangle (\strippadding+\stripwidth,\cardheight+0.2)}
\def\shapeRightStripShort{(\cardwidth-\stripwidth-\strippadding,\cardheight-\strippadding-1) rectangle (\cardwidth-\strippadding,\cardheight+0.2)}
\def\shapeTitleArea{(2*\strippadding+\stripwidth,\cardheight-\strippadding) rectangle (\cardwidth-2*\strippadding-\stripwidth,\cardheight-2*\stripwidth)}
\def\shapeContentArea{(2*\strippadding+\stripwidth,0.5*\cardheight) rectangle (\cardwidth+0.2,-0.2)}


%   Stylings f�r die Elemente definieren
\tikzset{
    %   runde Ecken f�r die Karten
    cardcorners/.style={
        rounded corners=0.2cm
    },
    %   runde Ecken f�r die "F�hnchen"
    elementcorners/.style={
        rounded corners=0.1cm
    }
}


\newcommand{\InvMoney}{\includegraphics[height = .6cm]{InvMoney}}
\newcommand{\Health}{\includegraphics[height = .6cm]{heart}}

%   Rahmen der Karte
\newcommand{\cardborder}{
    \draw[cardcorners] \shapeCard;
}

\newcommand{\cardhp}[1]{
		\node[left] at (\cardwidth-\stripwidth*.45,\cardheight-6.2*\stripwidth) {\textbf{\LARGE #1} \Health};
}
\newcommand{\cardtier}[1]{
		\node at (\cardwidth/8,\cardheight/8) {#1};
}

\newcommand{\modifier}{
\node[rotate=270] at (.5,6) {\textbf{\Large{\uppercase{Modifier}}}};
}
%   Hintergrund der Karte
\newcommand{\cardbackground}[1]{
		\clip[cardcorners] \shapeCard;
		%\draw[fill=white] (\cardwidth-.2,\cardheight-.2)--(.2,\cardheight-.2)--(.2,.2)--(\cardwidth-.2,.2)--cycle;
    %\node at (\cardwidth/2, \cardheight/2) {\includegraphics[height = 9.5cm]{#1}};
		\node[rotate=90] at (.5,6) {\textbf{\Large{\uppercase{#1}}}};
}

\newcommand{\modcardbackground}[1]{
		\clip[cardcorners] \shapeCard;
		%\draw[fill=white] (\cardwidth-.2,\cardheight-.2)--(.2,\cardheight-.2)--(.2,.2)--(\cardwidth-.2,.2)--cycle;
		\node at (\cardwidth-4*\stripwidth, \cardheight-\stripwidth) {\textbf{\Large{\uppercase{#1}}}};
}

\newcommand{\cardtitle}[1]{
    \node[text width=3cm, below right] at (2,8.3) {
       \shadowtext{\parbox{2.8cm}{\begin{spacing}{1.2}\textbf{\hspace{-.5cm}\uppercase{\Large #1}}\end{spacing}}}
    };
}
\newcommand{\modcardtitle}[1]{
    \node[text width=3cm, below right, rotate=270] at (\cardwidth*.95, \cardheight*.8) {
       \shadowtext{\parbox{2.8cm}{\begin{spacing}{1.2}\textbf{\hspace{-.5cm}\uppercase{\Large #1}}\end{spacing}}}
    };
}
\newcommand{\cardcontent}[1]{
    %\node[text width=4.5cm] at (\cardwidth/2, 0.35*\cardheight) {#1};
		\node[text width=4.4cm, below right] at (\stripwidth*1.1, 0.46*\cardheight) {\textrm{#1}};
}

\newcommand{\modcardcontent}[1]{
		\node[text width=3.5cm, below right, rotate=270] at (\cardwidth-\stripwidth*1.1, 0.46*\cardheight) {\textrm{#1}};
}

\newcommand{\cardimage}[1]{
	\node at (\cardwidth*.6, \cardheight*.65) {\includegraphics[height = 2.5cm]{#1}};
}


\newcommand{\modcardimage}[2]{
	\node[rotate=270] at (\cardwidth*.5, \cardheight*.65) {\includegraphics[height = 2.5cm]{#1}};
	\node at (.5, \cardheight-.5) {\includegraphics[height = .8cm]{#2}};
}


\newcommand{\player}[1]{
	\node at (.5, .5) {\includegraphics[height = .8cm]{#1}};
}

\newcommand{\cardprice}[1]{
		\node at (\stripwidth*1.7,\cardheight-1.2*\stripwidth) {\textbf{\LARGE #1}};
}
