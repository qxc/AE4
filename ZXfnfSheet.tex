\nonstopmode
\documentclass{article}
%   VORAUSGESETZTE LIBS
%   ------------------

%   R�nder des Dokuments anpassen
\usepackage[margin=6mm,top=5mm]{geometry}

%   Sharten Faken der aufen kaken
\usepackage[none]{hyphenat}

%   Schriftart der auf den Karten eingesetzten Texte
\usepackage{anttor}

%   UTF-8 Encoding der TeX-Dateien
\usepackage[utf8]{inputenc}

%   deutsches Sprachpaket
\usepackage[german]{babel}

%   optischer Randausgleich
\usepackage{microtype}

%   Einbinden von Grafiken
%\usepackage{graphicx}

%   Definieren und Verwenden von Farben
\usepackage{color}

%   TikZ zum "Malen" von Grafiken, in diesem Falle f�r die Karten
\usepackage{tikz}
\usetikzlibrary{patterns}
\usetikzlibrary{shadows}

%   Symbole dazuladen; Verwendung \ding{<nummer>}
\usepackage{pifont}
%   weitere Symbole
\usepackage{fourier-orns}
%\usepackage{textcomp}
%\usepackage{mathcomp}


\usepackage{DejaVuSansMono}
\renewcommand*\familydefault{\ttdefault} %% Only if the base font of the document is to be typewriter style
\usepackage[T1]{fontenc}

\usepackage{xcolor}
\usepackage{shadowtext}

\usepackage{setspace}
\usepackage{mwe}
%   FARBEN DER ELEMENTE/BESTANDTEILE DER KARTEN
%   -----------------------------------------

%   Hintergrundfarbe f�r den Titel-Kasten
    \definecolor{titlebg}{RGB}{30,30,30}

%   Farben der "F�hnchen" zur Kennzeichnung der unterschiedlichen Kartentypen
    \definecolor{defensebg}{RGB}{0,100,200}
    \definecolor{positivebg}{RGB}{80,180,0}
    \definecolor{negativebg}{RGB}{200,50,50}
    \definecolor{cannonbg}{RGB}{100,100,100}
		\definecolor{actionbg}{RGB}{0,0,255}
		\definecolor{attackbg}{RGB}{255,128,0}
		\definecolor{moneybg}{RGB}{230,180,0}
		%\definecolor{itembg}{RGB}{230,180,0}

%   Farbe des "F�hnchens" zur Angabe des Preises der Karten
    \definecolor{pricebg}{RGB}{230,180,0}

%   Hintergrundfarbe f�r den Textbereich
    %\definecolor{content}{RGB}{250,250,245}
    \definecolor{contentbg}{RGB}{255,255,255}
\usepackage[none]{hyphenat}
\pgfmathsetmacro{\cardwidth}{17.5}
\pgfmathsetmacro{\cardheight}{26}
\pgfmathsetmacro{\offset}{.5}
\pgfmathsetmacro{\tw}{8cm}
\pgfmathsetmacro{\squarewidth}{2.5}
\pgfmathsetmacro{\squareheight}{2.25}
\def\shapeCard{(0,0) rectangle (\cardwidth,\cardheight)}
\newcommand{\Money}{\includegraphics[height = .4cm]{money} }
\newcommand{\cardborder}{\draw[black] \shapeCard;
\draw[ultra thick] (\cardwidth/2, \cardheight/2) rectangle (0, \cardheight);}
\newcommand{\name}[1]{\node[text width = \tw] at (\cardwidth/4, \cardheight-\offset*2) {\LARGE{#1}};}
\newcommand{\setup}[1]{\node[text width = \tw] at (\cardwidth/4, \cardheight*.87) {\large \textbf{Setup:} #1};}
\newcommand{\unleash}[1]{\node[text width = \tw] at (\cardwidth/4, \cardheight*.67) {\large \textbf{Unleash:} #1};}
\newcommand{\difficulty}[1]{\node[text width = \tw] at (\cardwidth/4, \cardheight*.20) {\textbf{Additional Difficulty Rules:} #1};}
\newcommand{\rules}[1]{\node at (\cardwidth*.7, \cardheight/2+10) ; \node[text width = 7 cm] at (\cardwidth*3/4, \cardheight/2+2) {\large \textbf{Rules:} #1};}
\newcommand{\image}[1]{\includegraphics[height = .8cm]{#1}}
\newcommand{\hp}[1]{\textbf{#1} \image{heart}}
\newcommand{\diffR}[1]{\node[text width = \tw] at (\cardwidth*.9, \cardheight*.95) {\textbf{Difficulty Rating:} #1};}
\newcommand{\ability}[1]{\node[text width = \tw] at (\cardwidth/4, \cardheight*.7) {\large \textbf{Ability:} #1};}
\newcommand{\image}[1]{\includegraphics[height = .8cm]{#1}}
\newcommand{\type}[1]{#1}
\newcommand{\abilitycost}[1]{#1}
\newcommand{\sync}[1]{\node[text width = \tw] at (\cardwidth*.8, \cardheight*.2) {\large \textbf{Version:} #1};}
\begin{document}
\begin{center}
\pagestyle{empty}\begin{tikzpicture} 
\draw[black] \shapeCard; \newline\newcommand{\Or}{\textbf{OR}};
\draw[ultra thick] (\cardwidth/2, \cardheight*.56) rectangle (0, \cardheight); 
\name{Amalgamation \image{Foe1} \type{Foe}};
\setup{Place the Broodling cards above this mat.}
\rules{};\ability{\abilityName Call the Hive: \abilityCost{ 5}
Put minion from personal deck into play};

\end{tikzpicture}

\begin{tikzpicture} 
\draw[black] \shapeCard; \newline\newcommand{\Or}{\textbf{OR}};
\draw[ultra thick] (\cardwidth/2, \cardheight*.56) rectangle (0, \cardheight); 
\name{Time Sweeper \image{Foe2} \type{Foe}};
\setup{}
\rules{};\ability{\abilityName Call the Hive: \abilityCost{ 4}
Reveal the top 3 cards, put a minion revealed this way into play. Return the rest in any order. If no minion revealed, Foe gains 3 charges.};

\end{tikzpicture}

\begin{tikzpicture} 
\draw[black] \shapeCard; \newline\newcommand{\Or}{\textbf{OR}};
\draw[ultra thick] (\cardwidth/2, \cardheight*.56) rectangle (0, \cardheight); 
\name{Jeweled Wrath \image{Foe3} \type{Foe}};
\setup{Place the Cursed Shard cards above this mat to form the Cursed Shard supply pile.}
\rules{};\ability{\abilityName Jeweled Rot: \abilityCost{ 4}
The player with the fewest gems in hand gains a Cursed Shard, places it into their hand, and suffers 1 damage for each gem in hand.};

\end{tikzpicture}

\begin{tikzpicture} 
\draw[black] \shapeCard; \newline\newcommand{\Or}{\textbf{OR}};
\draw[ultra thick] (\cardwidth/2, \cardheight*.56) rectangle (0, \cardheight); 
\name{Glutton's Apprentice \image{Foe4} \type{Foe}};
\setup{At the start of the game, destroy all but 1 card in any supply pile. Foe gains 2 charges.}
\rules{If Foe would destroy a card in a supply pile and cannot, Gravehold suffers 3 damage instead.};\ability{\abilityName Bloody Conduit: \abilityCost{ 4}
Destroy three cards in any supply pile.};

\end{tikzpicture}

\begin{tikzpicture} 
\draw[black] \shapeCard; \newline\newcommand{\Or}{\textbf{OR}};
\draw[ultra thick] (\cardwidth/2, \cardheight*.56) rectangle (0, \cardheight); 
\name{Dalana, the Healer \image{Friend1} \type{Friend}};
\setup{No special setup.}
\rules{};\ability{\abilityName Bandage: \abilityCost{ 6}
Focus any player's breach. Each player gains 1 life for each opened breach they have.};

\end{tikzpicture}

\begin{tikzpicture} 
\draw[black] \shapeCard; \newline\newcommand{\Or}{\textbf{OR}};
\draw[ultra thick] (\cardwidth/2, \cardheight*.56) rectangle (0, \cardheight); 
\name{Sahala \image{Friend2} \type{Friend}};
\setup{}
\rules{};\ability{\abilityName Starlight Infusion: \abilityCost{ 4}
Place a token on a breach that doesn't have one. Spells cast from a breach with a token deals +2 damage.};

\end{tikzpicture}

\begin{tikzpicture} 
\draw[black] \shapeCard; \newline\newcommand{\Or}{\textbf{OR}};
\draw[ultra thick] (\cardwidth/2, \cardheight*.56) rectangle (0, \cardheight); 
\name{Qu, the badass \image{Friend3} \type{Friend}};
\setup{Token starts on build side}
\rules{Token says: DESTROYER token gives +2 damage to all spells cast that turn. BUILDER token gives +1 money whenever money is gained. At end of turn, flip token and return to this mat};\ability{\abilityName DESTROY or build: \abilityCost{ 4}
Give a player the token. };

\end{tikzpicture}

\begin{tikzpicture} 
\draw[black] \shapeCard; \newline\newcommand{\Or}{\textbf{OR}};
\draw[ultra thick] (\cardwidth/2, \cardheight*.56) rectangle (0, \cardheight); 
\name{Joanna, the Merchant \image{Friend4} \type{Friend}};
\setup{}
\rules{};\ability{\abilityName Barter: \abilityCost{ 5}
Shuffle any player's turn order card in the turn order discard pile into the turn order deck.};

\end{tikzpicture}


 \end{center} 
 \end{document}