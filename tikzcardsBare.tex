%   COMMANDS ZUM ZUSAMMENBAUEN DER KARTEN
%   ---------------------------------------

%   TikZ/PGF Settings f�r die Karten
\pgfmathsetmacro{\cardwidth}{6}
\pgfmathsetmacro{\cardheight}{8.5}
\pgfmathsetmacro{\imagewidth}{\cardwidth}
\pgfmathsetmacro{\imageheight}{0.75*\cardheight}
\pgfmathsetmacro{\stripwidth}{0.7}
\pgfmathsetmacro{\strippadding}{0.2}
\pgfmathsetmacro{\textpadding}{0.1}
\pgfmathsetmacro{\titley}{.94*\cardheight}
\pgfmathsetmacro{\titlex}{2.3}



%   Formen der einzelnen Kartenelemente/-bestandteile
\def\shapeCard{(0,0) rectangle (\cardwidth,\cardheight)}
\def\shapeLeftStripTop{(\strippadding,\cardheight/2+1) rectangle (\strippadding+\stripwidth,\cardheight-\strippadding-\strippadding-1)}
\def\shapeLeftStripBot{(\strippadding,-0.2) rectangle (\strippadding+\stripwidth,\cardheight/2-3)}
\def\shapeLeftStripShort{(\strippadding,\cardheight-\strippadding-1) rectangle (\strippadding+\stripwidth,\cardheight+0.2)}
\def\shapeRightStripShort{(\cardwidth-\stripwidth-\strippadding,\cardheight-\strippadding-1) rectangle (\cardwidth-\strippadding,\cardheight+0.2)}
\def\shapeTitleArea{(2*\strippadding+\stripwidth,\cardheight-\strippadding) rectangle (\cardwidth-2*\strippadding-\stripwidth,\cardheight-2*\stripwidth)}
\def\shapeContentArea{(2*\strippadding+\stripwidth,0.5*\cardheight) rectangle (\cardwidth+0.2,-0.2)}


%   Stylings f�r die Elemente definieren
\tikzset{
    %   runde Ecken f�r die Karten
    cardcorners/.style={
        rounded corners=0.2cm
    },
    %   runde Ecken f�r die "F�hnchen"
    elementcorners/.style={
        rounded corners=0.1cm
    }
}


\newcommand{\InvMoney}{\includegraphics[height = .6cm]{InvMoney}}
\newcommand{\Health}{\includegraphics[height = .6cm]{heart}}

%   Rahmen der Karte
\newcommand{\cardborder}{
    \draw[lightgray,cardcorners] \shapeCard;
}

%   Hintergrund der Karte
\newcommand{\cardbackground}[1]{
		\clip[cardcorners] \shapeCard;
		\draw[fill=white] (\cardwidth-.2,\cardheight-.2)--(.2,\cardheight-.2)--(.2,.2)--(\cardwidth-.2,.2)--cycle;
    %\node at (\cardwidth/2, \cardheight/2) {\includegraphics[height = 9.5cm]{#1}};
		\node[rotate=90] at (\stripwidth*1.7,\cardheight-4*\stripwidth) {\textbf{\Large{\uppercase{#1}}}};
}

\newcommand{\cardtitle}[1]{
    \node[text width=3cm, below right] at (\titlex,\titley) {
        
       \shadowtext{\parbox{2.8cm}{\begin{spacing}{1.2}\textbf{\hspace{-.5cm}\uppercase{\Large #1}}\end{spacing}}}
        
    };
}
\newcommand{\cardcontent}[1]{
    %\node[text width=4.5cm] at (\cardwidth/2, 0.35*\cardheight) {#1};
		\node[text width=4.4cm, below right] at (\stripwidth*1.1, 0.46*\cardheight) {\textrm{#1}};
}

\newcommand{\cardimage}[1]{
	\node at (\cardwidth*.6, \cardheight*.65) {\includegraphics[height = 2.5cm]{#1}};
}

\newcommand{\cardprice}[1]{
		%\pgfmathsetmacro{\x}{(2+#1)/12}
		%\pgfmathsetmacro{\redvalue}{255-(45*\x*\x)}
		%\pgfmathsetmacro{\greenvalue}{255-(85*\x*\x)}
		%\pgfmathsetmacro{\bluevalue}{255*(1-\x)*(1-\x)}
		%\pgfmathsetmacro{\x}{(#1)/10}
		%\pgfmathsetmacro{\redvalue}{(1-\x)*(1-\x)*0+2*\x*(1-\x)*275+\x*\x*200}
		%\pgfmathsetmacro{\greenvalue}{(1-\x)*(1-\x)*100+2*\x*(1-\x)*275+\x*\x*100}
		%\pgfmathsetmacro{\bluevalue}{(1-\x)*(1-\x)*200+2*\x*(1-\x)*275+\x*\x*0}
		\definecolor{currentcolor}{RGB}{0,0,0}
    \node at (\stripwidth*1.9,\cardheight-.8*\stripwidth) {\Money};
		\node at (\stripwidth*1.7,\cardheight-1.2*\stripwidth) {\color{currentcolor} \textbf{\LARGE #1}};
}

\newcommand{\cardhp}[1]{
		\node at (\cardwidth-\stripwidth*1.7,\cardheight-2.5*\stripwidth) {\textbf{\LARGE #1}\Health};
}